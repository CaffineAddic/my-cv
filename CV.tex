\documentclass[11pt]{article}
\usepackage[margin=1in]{geometry}
\usepackage{times}
\usepackage{enumitem}
\usepackage{fancyhdr}
\usepackage{comment}
\usepackage[colorlinks,urlcolor=blue]{hyperref}
\pagestyle{fancy}
\fancyhf{}
\fancyhead[C]{}
\fancyfoot[C]{\thepage}

% Define a new length variable for vspace
\newlength{\sectionvspace}
\setlength{\sectionvspace}{-5mm} % Set the value as needed

\newlength{\sectionvspacee}
\setlength{\sectionvspacee}{-3mm} % Set the value as needed


\begin{document}
	
	\begin{center}
		{\huge\bf Saumya Vilas Roy} \\
		
		\vspace{3mm} %5mm vertical space
		{\large
			+91 8826433226 |
			\href{mailto:saumyaroy@tutanota.com}{saumyaroy@tutanota.com} |
			\href{https://caffineaddic.github.io/}{Website} |
			New Delhi, India
		} \\
		{
			%	\small 
			%\href{https://www.github.com/CaffineAddic}{https://www.github.com/CaffineAddic}
			
		}
	\end{center}
		\vspace{\sectionvspacee}
	\vspace{\sectionvspace}
	\section*{\bf Summary}
	\vspace{\sectionvspace}
	\rule{\textwidth}{0.4pt}
	\begin{itemize}[leftmargin=*,noitemsep,topsep=0pt]
		\item Machine Learning/Deep (ML/DL) Learning researcher with 2.5 years of data analysis experience
		\item Skilled in ML/DL, Electronics, and Communication with an emphasis on biomedical data analysis
		\item Committed to developing innovative solutions in interdisciplinary projects
		\item Interested in ML/DL research opportunities in health and biomedical applications
	\end{itemize}
	
		\vspace{\sectionvspacee}
	\section*{\bf Education}
	\vspace{\sectionvspace}
	\rule{\textwidth}{0.4pt}
	\begin{itemize}[leftmargin=*,noitemsep,topsep=0pt]
		\item Bachelor of Technology in Electronics and Communication Engineering, \hfill \textbf{Nov 2020 - May 2024} \\
		Indian Institute of Space Science and Technology (IIST), Kerala \hfill \textbf{GPA: 3.12}
		\begin{itemize}[leftmargin=*,noitemsep,topsep=0pt]
			
			\item 	Advisors: Dr. Deepak Mishra (IIST), Dr. Rajesh Sadananan (IIST) and Dr. Satheesh K. (IIST) \\
			Developed a novel method for estimating non-uniform temperature profiles in combustion systems using Laser Absorption Spectroscopy (LAS) and Multi-Output Gaussian Process Regression.
			\item Scholarship from Department of Space, Govt. of India.
		\end{itemize}
		\item High School Diploma, XII (Central Board of Secondary Education), \hfill \textbf{2018 - 2020}\\
		Ryan International School, New Delhi   \hfill \textbf{Percentage: 90.6 \%}
	\end{itemize}
	
		\vspace{\sectionvspacee}
	\section*{\bf Research Experience}
	\vspace{\sectionvspace}
	\rule{\textwidth}{0.4pt}
	\begin{itemize}[leftmargin=*,noitemsep,topsep=0pt]
		
		\item Research Intern, \hfill \textbf{June 2024 - Current} \\
		Indian Institute of Technology (IIT), Delhi \\
		Advisors: Dr. Ankur Miglani (IIT, Indore) and Dr. Husain Kanchwala (IIT, Delhi)
		\begin{itemize}[leftmargin=*,noitemsep,topsep=0pt]
			\item Developed and implemented deep learning convolutional neural networks (CNNs) to detect damage on high-magnification images of wheat grain kernels.
			\item Designed and deployed an AI-driven safety edge device (esp32) to prevent accidents in construction environments by detecting and alerting on unsafe behavior.
		\end{itemize}
		
		
\item Summer Intern, \hfill \textbf{May 2023 - August 2023}\\
National Remote Sensing Center (NRSC), Indian Space Research Organization (ISRO)\\
Advisors: Dr. Mishra and Ms. Haripriya S. (NRSC)
\begin{itemize}[leftmargin=*,noitemsep,topsep=0pt]
	\item Developed and applied a U-net Complex Valued Neural Network for segmenting raw PolSAR images using the Pauli representation.
	\item Analyzed the effects of different dropout rates on model overfitting and enable raw processing of PolSAR image without domain shift.
\end{itemize}
		
\begin{comment}
			\item Undergraduate Researcher, \hfill \textbf{Aug 2021 - May 2024} \\
	Indian Institute of Space Science and Technology
	\begin{itemize}[leftmargin=*,noitemsep,topsep=0pt]
		\item Collaborated with Dr. Marcos M. Raimundo (University of Campinas, Brazil) and Dr. Mishra to develop a semi-supervised learning approach with spatial transformers for medical image registration, utilizing a hybrid dataset of real and synthetic images to reduce training data requirements while leveraging transfer learning to curtail computational overhead.	
		\item Created and validated a Schlieren/RGB Flame Images Analyzing Tool based on Fast Fourier Transform (FFT) and Wavelet Transform to analyze time-series flame images to identify the region of instability and the corresponding oscillating frequency in collaboration with Dr. Rajesh
		Sadananan (IIST).
		\item Collaborated with Dr. Manoj B.S. (IIST) on a Complex Network Analysis project, focusing on the OPEC Crude Oil Trade Network. Utilized graph theory to model global crude oil flows between nations, identifying key time-series trends and predicting potential fluctuations in price and demand.
	\end{itemize}
	\end{itemize}
\end{comment}
	
	\item Undergraduate Researcher, \hfill \textbf{Aug 2021 - May 2024}
	
	Indian Institute of Space Science and Technology
	
	\begin{itemize}[leftmargin=*,noitemsep,topsep=0pt]
		\item Advisors: Dr. Marcos M. Raimundo (University of Campinas, Brazil) and Dr. Mishra

		
Developed a semi-supervised learning approach with spatial transformers for medical image registration, utilizing a hybrid dataset of real and synthetic images to reduce training data requirements.
		\item Advisors: Dr. Sadananan and Dr. Mishra
		
Created a Schlieren/RGB Flame Images Analyzing Tool using FFT and Wavelet Transform to analyze time-series flame images and identify instability regions and oscillating frequencies.
		\item Advisor: Dr. Manoj B.S. (IIST) 
		
Utilized graph theory to model global crude oil flows between nations, identifying key time-series trends and predicting potential fluctuations in price and demand accurately over time.
		
	\end{itemize}
	
	
		\section*{\bf First-Author Publications}
	\vspace{\sectionvspace}
	\rule{\textwidth}{0.4pt}
	\begin{itemize}[leftmargin=*,noitemsep,topsep=0pt]
		
		
		\item \textbf{Saumya Vilas Roy*}, Husain Kanchwala \& Ankur Miglani. Deep CNN-based damage classification of milled wheat grains using a high-magnification image dataset. (Manuscript in preparation).
		
		\item  \textbf{Saumya Vilas Roy*}, Deepak Mishra \& Marcos M. Raimundo. HybridMorph: Bridging the Gap between Synthetic and Real Data for Accurate MR Image Registration. (Manuscript in preparation).
		
		\item \textbf{Saumya Vilas Roy*}, Deepak Mishra, Satheesh K. \& Rajesh Sadananan. Estimating Non-Uniform Temperature Profiles in Combustion Systems using Laser Absorption Spectroscopy and Multi-Output Gaussian Process Regression. (Manuscript in preparation).
		
		\item \textbf{Saumya Vilas Roy*}, Deepak Mishra \& Rajesh Sadananan (2025). Combined FFT and Wavelet Analysis of Schlieren and Flame Luminosity Time-Series to Visualize Regions of Combustion Instability. (Accepted NAPC 2025).
		
		\item \textbf{Saumya Vilas Roy*}, \& Manoj BS. (2024). A Complex Network Analysis of the OPEC Crude Oil Trade Network. \href{https://doi.org/10.36227/techrxiv.171169316.66809297/v2}{DOI: 10.36227/techrxiv.171169316.66809297/v2}. (RAICS 2024).
		
	\end{itemize}
	
	\vspace{\sectionvspacee}
\section*{\bf Skills}
\vspace{\sectionvspace}
\rule{\textwidth}{0.4pt}
\begin{itemize}[leftmargin=*,noitemsep,topsep=0pt]
	\item \textbf{Languages:} Python, C++, MATLAB, JavaScript, HTML/CSS, SQL.
	\item \textbf{Developer Tools:} Git, GNU Octave, LaTeX, AWS.
	\item \textbf{Libraries:} TensorFlow, PyTorch, Keras, OpenCV.
\end{itemize}


\vspace{\sectionvspacee}
\section*{\bf Awards/Recognition}
\vspace{\sectionvspace}
\rule{\textwidth}{0.4pt}
\begin{itemize}[leftmargin=*,noitemsep,topsep=0pt]
	\item \textbf{3rd} position in student's flash talks at Frontiers symposium in Data science 2024, IISER Trivandrum.
	\item Top \textbf{2\%} in the Joint Entrance Examination (JEE) Main and Advanced, a highly competitive national-level engineering entrance examination in India.
	\item \textbf{1st} position in Tinker Fest 2018 organized by ATAL tinkering labs for the project "Algae Based Air Purifier and Quality Sensor" at Ryan International School.
\end{itemize}
		\vspace{\sectionvspacee}

	
	\vspace{\sectionvspacee}	
\section*{\bf Presentations}
\vspace{\sectionvspace}% Adjust the value as needed
\rule{\textwidth}{0.4pt}
\begin{itemize}[leftmargin=*,noitemsep,topsep=0pt]
	\item "Complex Valued U-Net for Segmentation of PolSAR Images", ISG-ISRS 2023.
	\item "Meta-Learning for Space Applications for Advancements in Space Technology", Hindi Technical Conference 2023, IIST organized by Indian Space Research Organization (ISRO).
\end{itemize}

	\vspace{\sectionvspacee}
\section*{\bf References}
\vspace{\sectionvspace}% Adjust the value as needed
\rule{\textwidth}{0.4pt}
\begin{itemize}[noitemsep,topsep=0pt]
	\item \textbf{Husain Kanchwala}
	\begin{itemize}[noitemsep,topsep=0pt]
		\item \textbf{Title:} Assistant Professor, Center for Automotive Research and Tribology, IIT Delhi, India
		\item \textbf{Email:} \href{mailto:husaink@iitd.ac.in}{husaink@iitd.ac.in}
		\item \textbf{Phone:} +91-112-6548571
	\end{itemize}
	
	\item \textbf{Deepak Mishra}
	\begin{itemize}[noitemsep,topsep=0pt]
		\item \textbf{Title:} Professor, Department of Avionics, IIST, India
		\item \textbf{Email:} \href{mailto:deepak.mishra@iist.ac.in}{deepak.mishra@iist.ac.in}
		\item \textbf{Phone:} +91-471-2568583
	\end{itemize}
	
	\item \textbf{Marcos M. Raimundo}
	\begin{itemize}[noitemsep,topsep=0pt]
		\item \textbf{Title:} Assistant Professor, Institute of Computing, University of Campinas, Brazil
		\item \textbf{Email:} \href{mailto:mraimundo@ic.unicamp.br}{mraimundo@ic.unicamp.br}
		\item \textbf{Phone:} +55-19-35210322
	\end{itemize}
\end{itemize}
	
\end{document}